\section{Giới thiệu}
\label{sec:introduction}

\subsection{Tổng quan về bài toán phát hiện đối tượng}

Phát hiện đối tượng (Object Detection) được coi là một trong những lĩnh vực quan trọng nhất trong nghiên cứu máy học nói chung và xử lý ánh nói riêng. Các mô hình phát hiện đối tượng được training nhằm nhận diện các đối tượng, phát hiện đối tượng là gì trong các hình ảnh hoặc video.

Convolutional Neural Network(CNN) là một kiến trúc học sâu lấy cảm hứng từ nhận thức trực quan của sinh vật. Năm 1980 Kunihiko Fukushima\cite{fukushima1980self} đề xuất neocognitron , có thể coi là tiền thân của CNN. Kể từ đó, nhiều nghiên cứu mới về CNN đã được công bố và đưa ra nhiều thuật toán phát hiện đối tượng mạnh mẽ.

Một số ứng dụng phổ biến trong phát hiện đối tượng bao gồm:
\begin{itemize}
    \item Nhận dạng biển số. Tính năng phát hiện đối tượng được sử dụng thông qua việc lưu giữ lại hình ảnh và phát hiện các đối tượng cụ thể như xe cộ, phương tiện đi lại trên bức ảnh đó.
    \item Phát hiện và nhận dạng khuôn mặt: phát hiện khuôn mặt trong hình ảnh hoặc ai trong hình ảnh đó.
    \item Theo dõi đối tượng: công nghệ này có thể ứng dụng để theo dõi chuyển động của một đối tượng hay đồ vật cụ thể.
    \item Ô tô tự lái: sử dụng nhằm phát hiện từng đối tượng xung quanh xe để tránh va chạm, tuân thủ luật giao thông.
\end{itemize}

Các phần dưới, học viên tiến hành sử dụng các API mà không sử dụng cuda.

\subsection{TensorFlow Object Detection API}

\textbf{TensorFlow Object Detection API} là một open-source framework để xây dựng các mô hình object detection và image segmentation có thể localize nhiều đối tượng trong cùng một hình ảnh. Framework hỗ trợ cả 2 phiên bản tensorflow nhưng được khuyến khích dùng version 2.

Nhằm minh họa phần sử dụng API này, học viên sử dụng document của Tensorflow\cite{tensorflowapi}. Code sử dụng được trình bày trong \cite{tensorflowcode}.

Trong ví dụ minh họa này, hình ảnh được download từ internet thông qua url, đưa vào local API của Tensoflow để xử lý, kết quả được hiển thị như hình \ref{fig:tensor_result}. Trong đó bên trái là hình ảnh mẫu từ tài hướng dẫn, hinh bên phải là hình lấy từ internet, thời gian xử lý lần lượt là 28.4 và 30.3 giây.

\begin{figure}
    \centering
    \includegraphics[scale=0.25]{../code/processed/processed.jpg}\includegraphics[scale=0.25]{../code/processed/my_image.jpg}
    \caption{Kết quả của Tensorflow API}
    \label{fig:tensor_result}
\end{figure}

\subsection{Detectron2}

\textbf{Detectron2}\cite{wu2019detectron2} là một thư viện mã nguồn mở do Facebook AI research phát triển, hỗ trợ cả hai thuật toán detection và segmentation.

Sau quá trình cài đặt thử nghiệm, nhận thấy việc cài đặt tại máy khó khăn và cần nhiều thư viện tương thích, vì vậy sử dụng notebook mẫu của Meta và thay đổi đường dẫn hình ảnh. Code được thể hiện như trong \cite{detectron2demo}.

Sử dụng hình ảnh tùy chọn tương tự ví dụ với \textit{TensorFlow}, ta được kết quả như hình \ref{fig:detection2}

\begin{figure}
    \centering
    \includegraphics[scale=0.2]{../code/processed/detection2.png}
    \caption{Kết quả của Detectron2}
    \label{fig:detection2}
\end{figure}

\subsection{MXNet GluonCV}

\textbf{MXNet GluonCV} là một framework đang được phát triển bởi Apache Software Foundation \cite{apacheglucon,gluoncvnlp2020}. Nó hỗ trợ nhiều bài toán trong xử lý ảnh như Image Classification, Object Detection, Semantic Segmentation, Instance Segmentation\dots Vì vậy, ngoài các ứng dụng của nó, lượng tài liệu tham khảo và các ví dụ cũng được trình bày rất chi tiết và dễ hiểu.

Vì vậy, trong \cite{gluonCVcode}, học viên sẽ trình bày các ứng dụng cùng các mô hình có hỗ trợ.

Đối với bài toán Object Detection, có 3 mô hình là SSD, Faster RCNN và YOLO có ví dụ, kết quả predict theo thứ tự từ trái sang được thể hiện trong hình \ref{fig:gluoncv-detection}

\begin{figure}
    \centering
    \includegraphics[scale=0.35]{../code/processed/gluon_ssd.png}\includegraphics[scale=0.35]{../code/processed/gluon_faster_rcnn.png}\includegraphics[scale=0.35]{../code/processed/gluon_yolo3.png}
    \caption{Kết quả Object Detection của GluonCV}
    \label{fig:gluoncv-detection}
\end{figure}

Đối với bài toán Segmentation GluonCV hỗ trợ cả instance và semantic, kết quả pre-trained predict thể hiện trong hình \ref{fig:gluon_segmentation}. 

\begin{figure}
    \centering
    \includegraphics[scale=0.3]{../code/processed/gluon_segmentation.png}
    \includegraphics[scale=0.2]{../code/processed/FCN_sematic.png}
    \caption{Kết quả Segmentation của GluonCV}
    \label{fig:gluon_segmentation}
\end{figure}

Do giới hạn về thời gian, các ứng dụng khác của GluonCV không được hiện thực hết, có thể tham khảo thêm tại \cite{gluonCVdocument}.

\subsection{MMDetection}

MMDetection là một opensource framework hỗ trợ bài toán object detection được phát triển bởi OpenMMLab\cite{OpenMMLab}.

So với các thư viện trước, MMDetection khá dễ cài đặt và sử dụng, chỉ cần theo hướng dẫn tại document và chạy thử nghiệm, kết quả trả về cho ta hình ảnh predict như hình \ref{fig:MMDetection}

\begin{figure}
    \centering
    \includegraphics[scale=0.3]{../code/MMDetection/mmdetection/outputs/vis/tmpb1lbm2dd.jpg}
    \caption{Kết quả detection của MMDetection}
    \label{fig:MMDetection}
\end{figure}
