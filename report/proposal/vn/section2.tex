\section{Giới thiệu}
\label{sec:introduction}

\subsection{Tổng quan về bài toán phát hiện đối tượng}

Phát hiện đối tượng (Object Detection) được coi là một trong những lĩnh vực quan trọng nhất trong nghiên cứu máy học nói chung và xử lý ánh nói riêng. Các mô hình phát hiện đối tượng được training nhằm nhận diện các đối tượng, phát hiện đối tượng là gì trong các hình ảnh hoặc video.

Convolutional Neural Network(CNN) là một kiến trúc học sâu lấy cảm hứng từ nhận thức trực quan của sinh vật. Năm 1980 Kunihiko Fukushima\cite{fukushima1980self} đề xuất neocognitron , có thể coi là tiền thân của CNN. Kể từ đó, nhiều nghiên cứu mới về CNN đã được công bố và đưa ra nhiều thuật toán phát hiện đối tượng mạnh mẽ.

Một số ứng dụng phổ biến trong phát hiện đối tượng bao gồm:
\begin{itemize}
    \item Nhận dạng biển số. Tính năng phát hiện đối tượng được sử dụng thông qua việc lưu giữ lại hình ảnh và phát hiện các đối tượng cụ thể như xe cộ, phương tiện đi lại trên bức ảnh đó.
    \item Phát hiện và nhận dạng khuôn mặt: phát hiện khuôn mặt trong hình ảnh hoặc ai trong hình ảnh đó.
    \item Theo dõi đối tượng: công nghệ này có thể ứng dụng để theo dõi chuyển động của một đối tượng hay đồ vật cụ thể.
    \item Ô tô tự lái: sử dụng nhằm phát hiện từng đối tượng xung quanh xe để tránh va chạm, tuân thủ luật giao thông.
\end{itemize}
