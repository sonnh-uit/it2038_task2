\section{Tóm tắt đề tài}
\label{summary}

Trong báo cáo này, học viên trình bày một số nội dung cơ bản, phương pháp sử dụng và một số ví dụ của các thư viện được áp dụng trong bài toán phát hiện đối tượng. Các nội dung được trình bày sẽ bao gồm nhưng không giới hạn bởi khái niệm, phương pháp huấn luyện, phương pháp dự đoán, các cài đặt và các nội dung khác. Do giới hạn về tài nguyên và thời gian, các nội dung training trong các công cụ không được đề cập trong báo cáo này. Khi so sánh về hiệu suất huấn luyện của các công cụ, các kết quả từ các nghiên cứu đã được công bố sẽ được sử dụng thay vì kết quả thực nghiệm của học viên.

Phần \ref{sec:introduction} giới thiệu tổng quan về bài toán phát hiện đối tượng và các thư viện TensorFlow Object Detection API, Detectron2, MXNet GluonCV, MMDetectio cùng các ví dụ của chúng. Phần \ref{sec:compare} so sánh các thư viện với nhau và đưa ra kết luận. Cuối cùng là tài liệu tham khảo